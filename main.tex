% -*- coding: utf-8 -*-
%%
%%
%%
%%
%%
%%
%%  本模板可以使用以下两种方式编译:
%%
%%     1. PDFLaTeX
%%
%%     2. XeLaTeX [推荐]
%%
%%  注意:
%%    1. 在改变编译方式前应先删除 *.toc 和 *.aux 文件,
%%       因为不同编译方式产生的辅助文件格式可能并不相同。
%%
%%
\documentclass[12pt,openany]{book}
\usepackage{amsmath,amssymb}
\usepackage{ifxetex}
\ifxetex
  \usepackage[bookmarksnumbered]{hyperref}
\else
  \usepackage[unicode,bookmarksnumbered]{hyperref}
\fi

\usepackage[emptydoublepage]{NKThesis}   % 中文
%\usepackage[emptydoublepage,English]{NKThesis} % 英文

%   根据需要选择 biblatex 宏包选项.
\usepackage[backend = bibtex8, defernumbers = true,  sorting=nty,  style = nkthesis]{biblatex}
%\usepackage[numbers,sort&compress]{natbib} 
%\renewcommand\bibname{References}

\hypersetup{colorlinks=true,
            pdfborder=0 0 1,
            citecolor=black,
            linkcolor=black}
\usepackage{tikz}

\usepackage{pdfpages}

%%%%下面是三种参考文献格式
%%第一种,如果换成前两种在reference.tex中需要变为\bibliography{nkthesis},隐藏别的
%\bibliographystyle{abbrv}

%%%第二种,2015-author-year推荐使用
%\bibliographystyle{gbt-7714-2015-author-year}
%\usepackage[numbers, sort&compress]{natbib}
%\renewcommand\bibname{References}

%%%第三种
\addbibresource{nkthesis.bib}
\DeclareBibliographyCategory{cited}
\AtEveryCitekey{\addtocategory{cited}{\thefield{entrykey}}}
%\bibliographystyle{zy2}

\includeonly{
	abstract,
	manual,
	tikz,
	acknowledgements,
	references,
	appendices,
	resume
}
\newtheorem{Theorem}{\hskip 2em 定理}[chapter]
\newtheorem{Lemma}[Theorem]{\hskip 2em 引理}
\newtheorem{Corollary}[Theorem]{\hskip 2em 推论}
\newtheorem{Proposition}[Theorem]{\hskip 2em 命题}
\newtheorem{Definition}[Theorem]{\hskip 2em 定义}
\newtheorem{Example}[Theorem]{\hskip 2em 例}
\newtheorem{Remark}[Theorem]{\hskip 2em 注}

\begin{document}

%  设置基本信息
%  注意:  逗号`,'是项目分隔符. 如果某一项的值出现逗号, 应放在花括号内, 如 {,}
%
\NKTsetup{%
  论文题目(中文) =现代绿色化学中的物理有机问题,
  副标题        = ——酵母菌催化反应机理和咪唑类离子液体酸度的研究,
  论文题目(英文) = Physical Organic Concerns in Green Chemistry: Mechanistic
  Aspects of Baker's Yeast Mediated Reduction and
  Measurements of Acidity of Imidazolium Ionic Liquids,
  论文作者       = ,
  学号           = ,
  指导教师       = ,
  指导教师职称   = ,
  申请学位       = ,
  培养单位       = ,
  学科专业       = ,
  研究方向       = ,
  答辩委员会主席  = ,
  评阅人       = ,
  中图分类号     = ,
  UDC            = ,
  学校代码       = 10055,
  密级           = 公开,
                   % 公开 | 限制 | 秘密 | 机密, 若为公开, 不填以下三项
  保密期限       = ,
  审批表编号     = ,
  批准日期       = ,
  论文完成时间   = 二〇二三年五月,
  答辩日期       = ,
  签字日期       = 2023\quad 年\quad 5 \quad 月\quad 22 \quad 日,
  论文类别       = 学历硕士,
                   % 博士 | 学历硕士 | 硕士专业学位 | 高校教师 | 同等学力硕士
  院/系/所       = ,
  专业           = ,
  联系电话       = ,
  Email          = ,
  通讯地址(邮编) = ,
  备注           = 
}


%%%%%下面是匿名版本,要先在fengmian.doc中将标题等改为自己论文的标题,再生成fengmian.pdf替换里边的fengmian.pdf

%%%%%%%%%%%%%%%%%%%%%%%%%%%%%%%%%下面三行是生成匿名版本
%%%%%%%%%%%%%%%%%%%%%%%%%%%%%%%%使用前需要注释掉\NKTsetup
%%%%%%%%%%%%%%%%%%%%%%%%%%%%%%%%如果用正常版本就注释下面三行取消\NKTsetup的注释
  %\includepdf{fengmian.pdf}
%\setcounter{page}{1}
 % \pagestyle{centerheadings}
%%%%%%%%%%%%%%%%%%%%%%%%%%%%%%%%%



\include{abstract}
\tableofcontents
\include{manual}
% -*- coding: utf-8 -*-


%\chapter{The Tikz Package}




% -*- coding: utf-8 -*-

\def\bibrangedash{ $\sim$ }
\printbibliography [ category = cited]

%\bibliographystyle{zy2}
%\bibliography{nkthesis}


%%%%如果出现参考文献编号为0,则需要删除.bbl文件后再编译
\include{acknowledgements}
\include{appendices}
\include{resume}

\end{document}
